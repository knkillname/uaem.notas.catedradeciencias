% ### Definición de terminología nueva.
\makeatletter
\newcommand{\@terminology}[2][]{\index{#1}{\boldmath \bfseries #2}}
\newcommand\terminology{\@dblarg\@terminology}
\makeatother

% ### Entorno para hacer divagaciones
\newcommand{\digresswidth}{0.985\textwidth}
\colorlet{digresscolor}{Linen}
\newsavebox{\digressbox}
\newenvironment{digress}[1]{
    \begin{lrbox}{\digressbox}\begin{minipage}{\digresswidth}
            {\subsubsection*{{\normalfont\emoji{mag}} #1}}}
            {\end{minipage}\end{lrbox}
    \bigskip\par\noindent%
    \colorbox{digresscolor}{\usebox{\digressbox}}\bigskip}

% Matemáticas
% -----------
% ### Entornos matemáticos
\newtheoremstyle{plain}{}{}{}{}{\bfseries\sffamily}{:}{\newline}{}

\theoremstyle{plain}
\newtheorem{theorem}{Teorema}[chapter]
\newtheorem{lemma}[theorem]{Lema}
\newtheorem{proposition}[theorem]{Proposición}
\newtheorem{corollary}[theorem]{Corolario}

\newtheoremstyle{definition}{}{}{}{}{\bfseries\sffamily}{:}{ }{}
\theoremstyle{definition}
\newtheorem{definition}{Definición}[chapter]
\newtheorem{example}{Ejemplo}[chapter]
\newtheorem{problem}{Problema}[chapter]
\newtheorem{exercise}{Ejercicio}[chapter]

\newtheoremstyle{remark}{}{}{}{}{\sffamily\itshape}{:}{ }{}
\theoremstyle{remark}
\newtheorem*{remark}{Observación}
\newtheorem*{note}{Nota}

\renewcommand{\qedsymbol}{\blacksquare}

% ### Notación
\newcommand{\Verdadero}{\textsc{Verdadero}}
\newcommand{\Falso}{\textsc{Falso}}