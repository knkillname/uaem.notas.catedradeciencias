\chapter{Prefacio}
\label{cha:prefacio}

No tengo dudas de que el curso \emph{Cátedra de Ciencias} fue creado con el
propósito de que los estudiantes de la carrera en Ciencias puedan presentar su
examen profesional, defendiendo un trabajo de investigación ante un jurado.
Los objetivos del curso se han centrado históricamente en la lectura y
presentación de artículos académicos, introduciendo así a los estudiantes en
una de las actividades más importantes de la vida académica: conocer y
comunicar los avances en el conocimiento.
Sin embargo, etimológicamente hablando, el \emph{catedrático} es el profesor
que ocupa la καθέδρα (\emph{kathédra}), es decir, el sillón desde el cual una
persona enseñaba en la antigua Grecia.
En la actualidad la cátedra se refiere a la posición de un profesor
universitario, y por extensión, a la clase que imparte.

En este curso quizá no aprenderás a ser un catedrático de sillón, ni tampoco un
divulgador de la ciencia, pero al menos espero que adquieras las herramientas
básicas para apreciar, discutir y comunicar los avances en el conocimiento
científico.
Para ello se te va a pedir que asistas a seminarios de investigación o
congresos académicos, que leas artículos científicos, y que presentes uno de
ellos ante tus compañeros, practicando en varias ocasiones para que puedas
mejorar tu desempeño en el arte de la presentación.

Las notas están organizadas como sigue:

\begin{itemize}
      \item \autoref{cha:filosofiaciencia}. Empezamos con un poco de filosofía de
            la ciencia para que amplíes tu perspectiva sobre la ciencia y puedas
            apreciar mejor el trabajo de los científicos.
      \item \autoref{cha:literaturacientifica}. Aquí te daré algunos consejos
            para que puedas leer y analizar artículos académicos.
      \item \autoref{cha:presentaciones}. El siguiente paso es presentar estos
            artículos, así que usaré esta sección para explicarte por qué tienes
            pánico escénico y cómo puedes superarlo, así como una guía de estilo
            básica para que puedas hacer presentaciones efectivas.
      \item \autoref{cha:modeloporcompetencias}. Finalmente, si quieres
            dedicarte a la cátedra, es importante que conozcas el modelo de
            aprendizaje por competencias, ya que es el que se usa en la mayoría
            de las universidades, incluyendo la nuestra.
\end{itemize}
