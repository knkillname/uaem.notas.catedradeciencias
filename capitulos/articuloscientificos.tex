\chapter{Literatura científica}
\label{cha:literaturacientifica}

En esta sección abordamos el tema de la lectura científica, que es una actividad
fundamental para cualquier investigador.
Al igual que cuando lees una novela, la lectura de un artículo académico
requiere de atención, concentración, y algo de imaginación, pero las similitudes
terminan ahí.
Los libros y artículos científicos son textos densos, con un lenguaje técnico y
especializado, y con una estructura muy particular.
En este capítulo te mostraré cómo se estructura un artículo científico y cómo
puedes leerlo de manera eficiente.

\section{Cómo surgieron los artículos científicos}
\label{sec:contexto}
Antes de ponernos a leer artículos académicos es importante entender cómo
surgieron y por qué son tan importantes… al menos eso fue lo que entendí luego
de afrontar las ideas de Khun en el capítulo anterior.

Desde que el ser humano fue capaz de llevar un registro escrito de sus
descubrimientos, la ciencia ha sido una actividad colectiva.
De acuerdo con Sócrates la escritura dogmatiza el conocimiento de manera que las
nuevas generaciones no tienen que esforzarse por aprender, sino que simplemente
deben leer.
Afortunadamente Platón decidió dejar registro de las enseñanzas de Sócrates y,
al contrario de lo que pensaba su maestro, la escritura ha sido una herramienta
fundamental para el desarrollo de la ciencia y las humanidades.
Sin estos registros escritos el conocimiento se habría perdido con el tiempo y
se tendría que redescubrir en cada generación, retrazando el avance de la
ciencia misma.
Sin embargo Platón tenía razón en un punto: los diálogos son una forma de
aprendizaje muy efectiva.

Muchas de las ideas que se encuentran plasmadas en libros y artículos académicos
hoy en día surgieron de discusiones entre colegas, principalmente en forma de
cartas.
Por ejemplo, Galileo solía intercambiar cartas con Kepler y con otros astrónomos
de su época donde discutían sus observaciones, teorías, e incluso la publicación
de las mismas.


\section{El proceso moderno de la publicación científica}
\label{sec:procesopublicacion}

\section{Cómo leer un artículo científico}
\label{sec:comoleer}

\section{Cómo escribir un artículo científico}
\label{sec:comoescribir}