\chapter{Literatura científica}
\label{cha:literaturacientifica}

En esta sección abordamos el tema de la lectura científica, que es una actividad
fundamental para cualquier investigador.
Al igual que cuando lees una novela, la lectura de un artículo académico
requiere de atención, concentración, y algo de imaginación, pero las similitudes
terminan ahí.
Los libros y artículos científicos son textos densos, con un lenguaje técnico y
especializado, y con una estructura muy particular.
En este capítulo te mostraré cómo se estructura un artículo científico y cómo
puedes leerlo de manera eficiente.

\section{Cómo surgieron los artículos científicos}
\label{sec:contexto}
Antes de ponernos a leer artículos académicos es importante entender cómo
surgieron y por qué son tan importantes… al menos eso fue lo que entendí luego
de afrontar las ideas de Khun en el capítulo anterior.

Desde que el ser humano fue capaz de llevar un registro escrito de sus
descubrimientos, la ciencia ha sido una actividad colectiva.
De acuerdo con Sócrates la escritura dogmatiza el conocimiento de manera que las
nuevas generaciones no tienen que esforzarse por aprender, sino que simplemente
deben leer.
Afortunadamente Platón decidió dejar registro de las enseñanzas de Sócrates y,
al contrario de lo que pensaba su maestro, la escritura ha sido una herramienta
fundamental para el desarrollo de la ciencia y las humanidades.
Sin estos registros escritos el conocimiento se habría perdido con el tiempo y
se tendría que redescubrir en cada generación.
Sin embargo Sócrates tenía razón en algo: los diálogos son una forma de
muy efectiva de presentar y discutir ideas.
Por ejemplo, Galileo solía intercambiar cartas con Kepler y con otros astrónomos
de su época donde discutían sus observaciones, teorías, e incluso la publicación
de sus descubrimientos en libros.

Durante la Edad Media y La Ilustración surgieron comunidades de intelectuales (o
\emph{philosophes}) que usaban cartas para discutir sus descubrimientos a
grandes distancias.
Una de estas era la \emph{República Literaria}, de la que se tiene registro
desde 1417 gracias a una carta del humanista Francesco Barbaro a su amigo Poggio
Bracciolini.
La imprenta fue adoptada rápidamente por este grupo, y revolucionó la forma en
que se distribuía el conocimiento a nivel mundial, consolidando una comunidad
científica global en el proceso.

Las primera revista científica \emph{Journal des sçavans} fue fundada en 1665
por Denis de Sallo en París gracias la iniciativa del ministro de finanzas
Jean-Baptiste Colbert.
Originalmente tenía el propósito explícito de publicar lo que acontecía en la
República Literaria, incluyendo un catálogo de libros, los obituarios de
intelectuales sobresalientes, los experimentos de la física y la química, y
finalmente las decisiones legales, edictos y mandatos de la iglesia y el
estado\cite{Fyfe2022-ca}.
El filósofo Henry Oldenburg compró el primer número de la revista y la llevó a
la \emph{Royal Society} de Londres, donde se convirtió en el modelo para la
publicación de la primera revista científica en inglés, \emph{Philosophical
    Transactions of the Royal Society}, fundada unas cuantas semanas después.
A diferencia de la revista francesa, Oldenburg decidió que no publicaría
noticias políticas o religiosas, sino asuntos relacionados con la filosofía
natural así como los experimentos llevados a cabo en la Royal Society.
En esta revista se publicaron los descubrimientos de Isaac Newton, Robert Hooke,
y otros científicos del siglo XVII.

La revista de Oldenburg es muy distinta a las revistas científicas actuales y
más parecida a un periódico con noticias de los experimentos y descubrimientos
de la época.
Oldenburg solicitaba a los lectores verificar los experimentos publicados en la
revista así como los rumores de nuevos descubrimientos.
Por ello se considera que esta fue la primera revista en implementar un sistema
de revisión por pares, aunque en la actualidad el sistema funciona al revés: la
revisión se hace antes de la publicación y no después.

\section{El proceso moderno de la publicación científica}
\label{sec:procesopublicacion}

\section{Cómo leer un artículo científico}
\label{sec:comoleer}

\section{Cómo escribir un artículo científico}
\label{sec:comoescribir}