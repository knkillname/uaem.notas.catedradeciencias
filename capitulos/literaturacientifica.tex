\chapter{Literatura científica}
\label{cha:literaturacientifica}

En esta sección abordamos el tema de la lectura científica, que es una actividad
fundamental para cualquier investigador.
Al igual que cuando lees una novela, la lectura de un artículo académico
requiere de atención, concentración, y algo de imaginación, pero las similitudes
terminan ahí.
Los libros y artículos científicos son textos densos, con un lenguaje técnico y
especializado, y con una estructura muy particular.
Aunque existen muchos tipos de textos científicos como libros, tesis, reportes
técnicos, patentes y memorias de congresos, en este capítulo centramos nuestra
atención en los artículos académicos, que son la forma más común de aprender
sobre los avances de vanguardia en un área del conocimiento.

\section{La primera revista académica}
\label{sec:contexto}
Antes de ponernos a leer artículos académicos es importante entender cómo
surgieron y por qué son tan importantes… al menos eso fue lo que entendí luego
de afrontar las ideas de Khun en el capítulo anterior.

Desde que el ser humano fue capaz de llevar un registro escrito de sus
descubrimientos, la ciencia ha sido una actividad colectiva.
De acuerdo con Sócrates la escritura dogmatiza el conocimiento de manera que las
nuevas generaciones no tienen que esforzarse por pensar, sino que simplemente
deben leer.
En cierto sentido Kuhn estaría de acuerdo con Sócrates, ya que la escritura
es el medio por el cuál el paradigma se transmite de una generación a otra.
Afortunadamente Platón decidió dejar registro de las enseñanzas de Sócrates y,
al contrario de lo que pensaba su maestro, el conocimiento se democratizó sin
impedir las revoluciones científicas.
No es el caso que cada generación tenga que reinventar los mismos conceptos una
y otra vez.
Sin embargo Sócrates tenía razón en algo: los diálogos son una forma de
muy efectiva de presentar y discutir ideas.
Por ejemplo, Galileo solía intercambiar cartas con Kepler y con otros astrónomos
de su época donde discutían sus observaciones, teorías, e incluso la publicación
de sus descubrimientos en libros.

Durante la Edad Media y La Ilustración surgieron comunidades de intelectuales (o
\emph{philosophes}) que usaban cartas para discutir sus descubrimientos a
grandes distancias.
Una de estas era la \emph{República Literaria}, de la que se tiene registro
desde 1417 gracias a una carta del humanista Francesco Barbaro a su amigo Poggio
Bracciolini.
La imprenta fue adoptada rápidamente por este grupo, y revolucionó la forma en
que se distribuía el conocimiento a nivel mundial, consolidando una comunidad
científica global en el proceso.

\begin{remember}
    La correspondencia entre científicos sentó las bases para la publicación
    académica moderna.
\end{remember}

Las primera revista científica \emph{Journal des sçavans} fue fundada en 1665
por Denis de Sallo en París gracias la iniciativa del ministro de finanzas
Jean-Baptiste Colbert.
Originalmente tenía el propósito explícito de publicar lo que acontecía en la
República Literaria, incluyendo un catálogo de libros, los obituarios de
intelectuales sobresalientes, los experimentos de la física y la química, y
finalmente las decisiones legales de la iglesia y el estado\cite{Fyfe2022-ca}.
El filósofo Henry Oldenburg consiguió el primer número de la revista y la
presentó a la \emph{Royal Society} de Londres, donde se convirtió en el modelo
para la publicación de la primera revista científica en inglés,
\emph{Philosophical Transactions of the Royal Society}, fundada unas cuantas
semanas después.
A diferencia de la revista francesa, Oldenburg decidió que no publicaría
noticias políticas o religiosas, sino asuntos relacionados con la filosofía
natural así como los experimentos llevados a cabo en la Royal Society.
En esta revista se publicaron los descubrimientos de Isaac Newton, Robert Hooke,
y otros científicos del siglo XVII.

La revista de Oldenburg es muy distinta a las revistas científicas actuales y
más parecida a un periódico con noticias de los experimentos y descubrimientos
de la época.
Oldenburg solicitaba a los lectores verificar los experimentos publicados en la
revista así como los rumores de nuevos descubrimientos, por lo que la revista
ya tenía los precedentes de la revisión por pares; esta es una de las
características más importantes de las revistas científicas actuales.

\section{El proceso de publicación científica}
\label{sec:procesopublicacion}
En la época de Oldenburg la publicación de un artículo científico era un proceso
directo y sencillo: el científico escribía una carta al editor de la revista
donde describía su descubrimiento (a veces en forma de monografía) y el editor
decidía si publicarla o no.
Este proceso ha cambiado mucho desde entonces.

Hoy en día los artículos académicos se distinguen de otros tipos de
publicaciones no sólo por su contenido, sino por el proceso de publicación.
Entender este proceso te ayudará a apreciar por qué los artículos científicos
tienen la calidad y el prestigio que tienen.

\subsection*{Tipos de publicaciones}

\begin{description}
    \item[Artículo de investigación] Presenta un descubrimiento original, un
        nuevo método, modelo, o teoría.
        A menudo estos artículos resuelven problemas muy especializados con un
        lenguaje altamente técnico.
        Cuando nos referimos a un \emph{artículo científico} en general, nos
        referimos concretamente a los de este tipo.
    \item[Artículo de revisión] Contiene un resumen de los avances recientes en
        un área del conocimiento.
        Son especialmente útiles para los científicos que quieren adentrarse en
        un área del conocimiento ajena a la suya.
    \item[Comunicación corta] Son pequeños reportes de descubrimientos en una
        investigación que todaía está en curso.
    \item[Carta al editor] Son comentarios o críticas a artículos publicados en
        la revista.
        A veces son una forma de discutir los resultados de un artículo
        publicado.
\end{description}

Tocaremos este tema con más detalle en la sección \ref{sec:comoescribir}.
A continuación hay que seleccionar cuidadosamente la revista a la que se enviará
el manuscrito: no todas las revistas son iguales, y cada una tiene un enfoque
diferente; algunas características que se deben considerar son:

\begin{description}
    \item[Área de conocimiento] Cada revista tiene un área del conocimiento en
        la que se especializa, como la biología o la física, o incluso una
        subárea como la biología molecular o la física de partículas.
    \item[Idioma] Aunque la mayoría de las revistas científicas están en inglés,
        hay revistas en otros idiomas como el español o el francés.
        Cabe mencionar que el inglés es el idioma \emph{de facto} de la
        ciencia.
        Publicar en otro idioma hoy por hoy limita la difusión del artículo.
    \item[Acceso abierto] Muchas revistas científicas cobran por el acceso a sus
        artículos (carísimo, por cierto).
        Sin embargo, en años recientes han surgido movimientos intelectuales
        que promueven al acceso abierto a la información científica, artística,
        y tecnológica bajo el principio de que el conocimiento es un bien
        público.
        Algunas revistas científicas solicitan a los autores que paguen por
        publicar sus artículos, lo que permite que los artículos sean de acceso
        abierto para cualquier lector.
    \item[Impacto] La calidad de una revista se mide por el número de veces que
        sus artículos son citados por otros científicos en sus propios trabajos.
        El \terminology{factor de impacto} se calcula dividiendo el número de
        citas que recibe la revista entre el número de artículos publicados en
        un periodo de tiempo.
        La idea de esta métrica es que una revista que publica artículos de alta
        calidad (creíbles, originales, y bien escritos) será citada con mayor
        frecuencia.
\end{description}

Asimismo, se debe tomar en cuenta de que hay diferentes tipos de artículos según
su propósito y contenido.

\subsection*{La revisión por pares}
Antes de publicar un artículo, este pasa por varias etapas de revisión que son
realizadas por otros colegas (llamados \emph{pares} en este contexto).
En principio, este proceso garantiza que los artículos publicados en la
revista sean de alta calidad.
Generalmente, las revistas de mayor prestigio y factor de impacto tienen un
proceso de revisión más riguroso.

La primera etapa de revisión es la evaluación por el editor, quien decide si el
artículo es relevante para la revista, si contribuye al conocimiento en el área,
y si cumple con las especificaciones técnicas de la revista como el formato y
la extensión.

En la segunda etapa se realiza una revisión externa, donde el editor envía el
manuscrito a dos o tres revisores expertos en el tema del artículo.
Esta revisión es anónima, es decir, los revisores no saben quién escribió el
artículo que están revisando, y el autor no sabe quiénes son los revisores.
Esto evita que los revisores tengan sesgos a favor o en contra del autor.
Al final del proceso los revisores envían un informe al editor con sus
comentarios y recomendaciones; aquí se decide si el artículo se publica, si
necesita modificaciones, o si se rechaza por completo.

Cuando un artículo requiere modificaciones da inicio la tercera etapa de
revisión.
Aquí el autor y los revisores trabajan juntos de manera iterativa para mejorar
el artículo hasta que esté listo para su publicación.

\begin{remember}
    El proceso de \terminology{revisión por pares} permite filtrar y mejorar los
    artículos científicos antes de su publicación.
    Este proceso facilita que los artículos publicados en una revista sean de
    alta calidad.
\end{remember}

Me gustaría decir que este proceso es perfecto, pero en la práctica no lo es.
A veces los revisores no entienden el artículo, exigen cambios innecesarios, o
incluso rechazan el artículo por razones personales.
Entre los cambios más comunes que piden los revisores están la adición de
referencias a artículos de los mismos revisores, sacando provecho de su posición
de poder.
Más aún, cuando el artículo aborda temas hiperespecializados que solo un puñado
de científicos en el mundo entienden, es difícil para el editor encontrar
revisores que sean expertos en el tema sin que haya un conflicto de interés o
sin que se comprometa el anonimato de las partes involucradas.

Un punto debatible es que el prestigio de la revista se suele medir con la tasa
de aceptación.
Por ejemplo, la prestigiosa revista \emph{Nature} tiene una tasa de aceptación
del 8\%, mientras que en \emph{Scientific Reports} es del 49\%.
Significa que \emph{Nature} rechaza el 92\% de los artículos que recibe, por lo
que en teoría los artículos publicados en \emph{Nature} son de mayor calidad que
los de \emph{Scientific Reports}.
Sin embargo, esto conduce a la preguntas obvias:
¿Cómo garantiza Nature que no ha rechabazado un artículo de alta calidad?
¿Cómo garantiza Scientific Reports que no ha publicado un artículo de baja
calidad?
En respuesta a este problema existen revistas como \emph{Journal of Trial and
    Error}\footurl{https://journal.trialanderror.org/}, donde se publican
artículos donde los mismos autores fracasan en obtener resultados
significativos, pero que son valiosos para la comunidad científica porque evitan
que otros cometan los mismos errores.
Asimismo, y a manera de broma triste, existe la revista \emph{Journal of
    Universal Rejection}\footurl{http://www.universalrejection.org/} donde todos
los artículos son rechazados sin importar su calidad, esto hace que la revista
sea una parodia de la revisión por pares e, incidentalmente, al tenera la tasa
de aceptación más baja (0\%), es la revista más \emph{prestigiosa} del mundo.

Asimismo, tanto el editor como los revisores son humanos que dependen de su
experiencia y conocimiento para evaluar un artículo.
Cuando un artículo presenta un descubrimiento aparentemente revolucionario, es
difícil para los revisores y el editor evaluar dicho descubrimiento.
Por ejemplo, en 1981 la entonces candidata a doctora Mary M. Tai inventó un
modelo matemático para determinar el área bajo las curvas metabólicas; su método
fue tan popular entre sus colegas del hospital donde ella trabajaba que lo
apodaron la \emph{fórmula de Tai}.
En 1994 Tai publicó su modelo en un artículo científico de la revista
\emph{Diabetes Care}\cite{Tai1994}, donde el editor, los revisores externos,
los colaboradores de la autora (uno de ellos de la propia universidad de Yale),
y la autora misma, no detectaron lo absurdo de la situación: la fórmula de Tai
era simplemente la regla del trapecio; un método de integración numérica que se
enseña en el primer curso de cálculo de preparatoria o universidad.
El método es conocido desde la antigua Babilonia, fue usado por Arquímedes para
calcular el área de un círculo, y por Newton y Leibniz para desarrollar el
cálculo diferencial e integral.
Es una herramienta fundamental en ciencias e ingeniería y lo puedes consultar en
cualquier libro básico de cálculo como \cite{Arizmendi2018}.
Al momento de escribir estas líneas el artículo de Tai tiene al menos 540 citas,
siendo la más reciente de inicios de 2024.

Sin embargo, a pesar de estos problemas el proceso de revisión por pares es el
mejor que tenemos para garantizar la calidad de los artículos científicos, y
este tipo de problemas son menores en comparación con los beneficios que aporta.

\subsection*{El caso Wakefield}
A veces el proceso de revisión por pares falla, y un artículo científico
deficiente es publicado con graves consecuencias para la sociedad.
En 1998 la revista \emph{Lancet} publicó un artículo del entonces médico Andrew
Wakefield que sugería una relación entre la vacuna triple viral y el autismo.
La noticia se difundió rápidamente, apareciendo en los medios de comunicación
de todo el mundo.
En Reino Unido, donde se llevó a cabo el estudio, la confianza en la vacuna
disminuyó rápidamente, lo que años más tarde resultó en un brote de sarampión
que afectó a miles de personas.

La revisión por pares no dectó anomalías en el estudio de Wakefield a pesar de
que la muestra era muy pequeña y no era estadísticamente significativa; más aún,
una afirmación tan seria como la de Wakefield requería de un estudio mucho más
riguroso.
Apenas un mes después de la publicación del artículo, múltiples estudios
incapaces de replicar los resultados de Wakefield refutaron su hipótesis; a
estos se les conoce como \emph{estudio de replicación}, y es una forma de
revisión por pares que ocurre posterior a la publicación.

El artículo y los subsecuentes trabajos de Wakefield siguieron causando polémica
en la sociedad en general durante años hasta que, en 2004, luego de una
investigación exhaustiva por parte del periodista Brian Deer, se desenmascaró el
fraude\cite{Deerc1127}.
Wakefield había sido contratado dos años antes por el abogado de un grupo
antivacunas que buscaba demandar a los fabricantes de la vacuna triple viral.
Wakefield alteró los datos del estudio seleccionando a 12 niños que ya tenían
autismo y haciendo que pareciera que la vacuna era la causa.
Se descubrió también que el artículo no contaba con la aprobación de un comité
de ética y que los datos habían sido alterados.
A consecuencia de esto diez de los coautores del artículo se retractaron y
finalmente, en 2010, el Consejo General de Medicina del Reino Unido retiró la
licencia de Wakefield al tiempo que la revista \emph{Lancet} retiró el artículo
de su sitio web.

El caso Wakefield es un ejemplo de cómo la revisión por pares no es perfecta y
su impacto negativo en la sociedad cuando no se lleva a cabo de manera
adecuada.
Varias enfermedades y lesiones permanentes podrían haberse evitado si el
artículo de Wakefield no hubiera sido publicado, y es un recordatorio de la
responsabilidad que tienen los científicos y editores al publicar un artículo.

\begin{remember}
    La revisión por pares tiene sus limitaciones y no está exenta de críticas.
    Las publicaciones académicas conlleban una gran responsabilidad para los
    autores y los editores, y pueden tener un impacto significativo en la
    sociedad si no se llevan a cabo con el debido respeto y rigor.
\end{remember}

\section{Cómo leer un artículo científico}
\label{sec:comoleer}

Si te encuentras en un posgrado o laboratorio, es probable que tengas que leer
artículos científicos con regularidad.
Ya que son textos densos y especializados deberías tener un objetivo claro al
leerlos.
Pregúnate a ti mismo qué quieres o necesitas aprender; ¿saber cómo funciona un
método concreto? ¿conocer los resultados de un estudio del cuál has escuchado
hablar? ¿conocer las novedades en tu área de estudio? O si alguien te ha
encargado leer un artículo, ¿qué es lo que esa persona espera que aprendas?

\subsection*{Dónde encontrar artículos científicos}
Si aún no tienes un artículo en mente, hay varias maneras de encontrar uno.
Existen bases de datos de artículos científicos donde puedes buscar artículos
por palabras clave, autor, o revista.
Generalmente tu institución educativa o laboral tiene acceso a estas bases de
datos a través de su biblioteca, por lo que no necesitas pagar por el acceso a
los artículos.

Google ofrece una herramienta orientada a la búsqueda de artículos científicos
llamada \emph{Google Scholar}\footurl{https://scholar.google.com/}.
Cada autor puede crear un perfil en Google Scholar donde se enlistan todos sus
artículos publicados, y cada artículo tiene un enlace a la versión en PDF o HTML
si está disponible.
Además, Google Scholar tiene herramientas para citar artículos, buscar artículos
relacionados, y crear alertas para recibir notificaciones cuando se publique un
artículo nuevo que cumpla con ciertos criterios de búsqueda.
En general, es una herramienta muy útil para la revisión de la literatura.

Si no tienes acceso institucional a las revistas científicas, hay varias
alternativas gracias al movimiento de acceso abierto.
El CONRICyT\footurl{https://www.conricyt.mx/}, por ejemplo, ofrece un catálogo de
revistas de acceso abierto a las que puedes acceder desde cualquier computadora.
Sólo tienes que acceder a la sección \emph{recursos abiertos} de su sitio web y
seleccionar la editorial o base de datos que te interese.

Una de las bases de datos más interesantes la patrocina la Universidad de
Cornell, se llama \emph{arXiv}\footurl{https://arxiv.org/}, pero es un caso
especial.
arXiv no es una revista ni tiene editoriales, sino que  es un repositorio de
manuscritos que pueden o no haber sido publicados en una revista científica.
En particular, no tiene un proceso de revisión por pares, no cobra por publicar
manuscritos, y tampoco cobra por acceder a ellos.
La única moderación que tiene arXiv es realizada por un equipo de voluntarios
que se encargan de verificar que los manuscritos cumplan con las normas de
arXiv, pero no verifican la calidad del contenido.
Celebridades de la ciencia como Stephen Hawking, Paul Ginsparg, y Juan Maldacena
han publicado manuscritos en arXiv.

\subsection*{Pasos para leer un artículo académico}

Leer un artículo científico como si fuera una novela es un despropósito: sólo
vas a aburrirte sin aprender nada.
El problema es que los artículos científicos están escritos en un lenguaje
técnico y especializado, y tratar de leerlos de corrido sin contexto ni
conocimientos previos es una receta para el desastre.
En lugar de eso, te recomiendo seguir estos pasos tomados de la guía de la
biblioteca de la \emph{National University}\cite{NationalUniversity2024}:

\begin{enumerate}
    \item \emph{Hojea el artículo}. Lo primero es invertir unos cuantos minutos
          a decidir si el artículo es relevante para ti.
          Lee el título, el resumen (\emph{abstract}), y las conclusiones que
          generalmente están al final del artículo.
          Quizás te llame la atención los encabezados de las secciones, las
          figuras, o las tablas.
          Esto te dará una idea general del contenido del artículo.
    \item \emph{Revisa el vocabulario}. Es normal que encuentres palabras que no
          entiendes; el lenguaje técnico y especializado es una barrera muy común
          para los lectores no especializados o que no hablan el idioma en el que
          está escrito el artículo (generalmente inglés).
          Subraya las palabras que no entiendes y busca su significado en un
          diccionario de la especialidad; también puedes buscar recursos
          didácticos en línea como videos o tutoriales.
          Si tu artículo contiene un listado de palabras clave, asegúrate de
          entenderlas todas antes de continuar.
    \item \emph{Explora el contenido}. Aunque no todos los artículos tienen la
          misma estructura, la mayoría sigue una variante de \emph{IMRyD}
          (\emph{Introducción, Métodos, Resultados y Discusión}):
          \begin{description}
              \item[Introducción] Generalmente el primer párrafo del artículo es
                  un resumen del mismo, y la primera sección introduce el tema
                  del artículo, su contexto, el problema que aborda, y la
                  importancia del mismo.
              \item[Métodos] Aquí el autor o autores describen cómo abordaron el
                  problema realizando una serie de acciones como son
                  experimentos, simulaciones, análisis estadísticos, revisiones
                  de la literatura, diseño de modelos, etc.
              \item[Resultados] Los métodos producen resultados, y aquí es donde
                  se plasman con tablas, gráficas, y descripciones detalladas,
                  pero sin interpretación, es decir, sin explicar qué
                  significado tienen.
              \item[Discusión] Discutir no significa pelear, sino analizar,
                  desmenuzar, y razonar acerca de los resultados y su
                  significado.
                  Las buenas discusiones incluyen comparaciones con otros
                  estudios, explicaciones de por qué los resultados son como
                  son, y sugerencias para futuras investigaciones.
          \end{description}
    \item \emph{Revisa las referencias}. Se dice que los científicos están
          \emph{parados sobre los hombros de gigantes} para referirse a que el
          conocimiento científico se construye sobre el conocimiento previo.
          Las referencias es la forma en que los autores reconocen el trabajo de
          otros científicos que les han precedido y reconocen la influencia de
          sus ideas en el trabajo propio.
          Si encuentras que hay un tema que te interesa o del que necesitas
          profundizar, revisa las referencias del artículo que estás leyendo.
          Incluso si no tienes duddas, revisar las referencias te dará una idea
          del contexto así como de los autores y revistas más influyentes en el
          tema.
    \item \emph{Reflexiona}. Al leer un artículo científico suelen surgir
          preguntas, dudas, y reflexiones.
          Algunas de ellas se resuelven más adelante en el artículo, pero otras
          no.
          Anota tus preguntas y dudas, y busca las respuestas en otros artículos
          o libros.
          Algunas preguntas básicas que puedes hacerte son:
          \begin{itemize}
              \item ¿Entiendo el problema que aborda el artículo así como su
                    importancia para el área?
              \item ¿Comprendo los términos y conceptos que utiliza el autor?
              \item ¿Dedico demasiado tiempo a entender un concepto o una
                    sección que quizás no sea tan relevante?
              \item ¿Esta investigación es relevante para mi trabajo o mis
                    intereses?
          \end{itemize}
    \item \emph{Repasa}. Es raro que entiendas un artículo científico a la
          primera lectura y de forma aislada.
          La mayoría de las ocasiones será necesario leer nuevamente el artículo
          y alguna que otra referencia.
          En la segunda lectura aparecerán detalles que no habías notado antes,
          te será más fácil entender los conceptos y los resultados, y notarás
          conexiones entre artículos o secciones que no habías notado antes.
          Si tienes tiempo, deja el artículo por un día o dos y vuelve a leerlo
          con una mente fresca.
\end{enumerate}

Estos pasos no tienen que seguirse al pie de la letra ni en el orden en que los
presento.
Generalmente tendrás que regresar a pasos anteriores, o incluso saltar algunos
pasos si ya tienes experiencia.
No te frustres si un artículo te toma mucho tiempo en leer, es normal; con la
práctica te volverás más eficiente.

\section{Cómo escribir un artículo académico}
\label{sec:comoescribir}
