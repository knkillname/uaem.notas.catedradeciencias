\chapter{Literatura científica}
\label{cha:literaturacientifica}

En esta sección abordamos el tema de la lectura científica, que es una actividad
fundamental para cualquier investigador.
Al igual que cuando lees una novela, la lectura de un artículo académico
requiere de atención, concentración, y algo de imaginación, pero las similitudes
terminan ahí.
Los libros y artículos científicos son textos densos, con un lenguaje técnico y
especializado, y con una estructura muy particular.
En este capítulo te mostraré cómo se estructura un artículo científico y cómo
puedes leerlo de manera eficiente.

\section{La primera revista académica}
\label{sec:contexto}
Antes de ponernos a leer artículos académicos es importante entender cómo
surgieron y por qué son tan importantes… al menos eso fue lo que entendí luego
de afrontar las ideas de Khun en el capítulo anterior.

Desde que el ser humano fue capaz de llevar un registro escrito de sus
descubrimientos, la ciencia ha sido una actividad colectiva.
De acuerdo con Sócrates la escritura dogmatiza el conocimiento de manera que las
nuevas generaciones no tienen que esforzarse por pensar, sino que simplemente
deben leer.
Afortunadamente Platón decidió dejar registro de las enseñanzas de Sócrates y,
al contrario de lo que pensaba su maestro, la escritura ha sido una herramienta
fundamental para el desarrollo de la ciencia y las humanidades.
Sin estos registros escritos el conocimiento se habría perdido con el tiempo y
se tendría que redescubrir en cada generación.
Sin embargo Sócrates tenía razón en algo: los diálogos son una forma de
muy efectiva de presentar y discutir ideas.
Por ejemplo, Galileo solía intercambiar cartas con Kepler y con otros astrónomos
de su época donde discutían sus observaciones, teorías, e incluso la publicación
de sus descubrimientos en libros.

Durante la Edad Media y La Ilustración surgieron comunidades de intelectuales (o
\emph{philosophes}) que usaban cartas para discutir sus descubrimientos a
grandes distancias.
Una de estas era la \emph{República Literaria}, de la que se tiene registro
desde 1417 gracias a una carta del humanista Francesco Barbaro a su amigo Poggio
Bracciolini.
La imprenta fue adoptada rápidamente por este grupo, y revolucionó la forma en
que se distribuía el conocimiento a nivel mundial, consolidando una comunidad
científica global en el proceso.

\begin{remember}
    La correspondencia entre científicos sentó las bases para la publicación
    académica moderna.
\end{remember}

Las primera revista científica \emph{Journal des sçavans} fue fundada en 1665
por Denis de Sallo en París gracias la iniciativa del ministro de finanzas
Jean-Baptiste Colbert.
Originalmente tenía el propósito explícito de publicar lo que acontecía en la
República Literaria, incluyendo un catálogo de libros, los obituarios de
intelectuales sobresalientes, los experimentos de la física y la química, y
finalmente las decisiones legales de la iglesia y el estado\cite{Fyfe2022-ca}.
El filósofo Henry Oldenburg consiguió el primer número de la revista y la
presentó a la \emph{Royal Society} de Londres, donde se convirtió en el modelo
para la publicación de la primera revista científica en inglés,
\emph{Philosophical Transactions of the Royal Society}, fundada unas cuantas
semanas después.
A diferencia de la revista francesa, Oldenburg decidió que no publicaría
noticias políticas o religiosas, sino asuntos relacionados con la filosofía
natural así como los experimentos llevados a cabo en la Royal Society.
En esta revista se publicaron los descubrimientos de Isaac Newton, Robert Hooke,
y otros científicos del siglo XVII.

La revista de Oldenburg es muy distinta a las revistas científicas actuales y
más parecida a un periódico con noticias de los experimentos y descubrimientos
de la época.
Oldenburg solicitaba a los lectores verificar los experimentos publicados en la
revista así como los rumores de nuevos descubrimientos, por lo que la revista
ya tenía los precedentes de la revisión por pares; esta es una de las
características más importantes de las revistas científicas actuales.

\section{El proceso de publicación científica}
\label{sec:procesopublicacion}
En la época de Oldenburg la publicación de un artículo científico era un proceso
directo y sencillo: el científico escribía una carta al editor de la revista
donde describía su descubrimiento (a veces en forma de monografía) y el editor
decidía si publicarla o no.
Este proceso ha cambiado mucho desde entonces.

% Existen varios tipos de publicaciones científicas, entre las que se encuentran
% las revistas, las actas de congresos, los libros, y las tesis.
% Así mismo, los artículos científicos pueden ser de diferentes tipos, dependiendo
% si son cuantitativos o cualitativos, si son teóricos o empíricos, si son
% revisión de la literatura o presentan un nuevo descubrimiento, etc.

Cuando un científico quiere publicar un artículo, primero debe escribir un
manuscrito que describa su descubrimiento.
Tocaremos este tema con más detalle en la sección \ref{sec:comoescribir}.

\subsection*{Tipos de revistas científicas}
A continuación hay que seleccionar cuidadosamente la revista a la que se enviará
el manuscrito: no todas las revistas son iguales, y cada una tiene un enfoque
diferente; algunas características que se deben considerar son:

\begin{description}
    \item[Área de conocimiento] Cada revista tiene un área del conocimiento en
        la que se especializa, como la biología o la física, o incluso una
        subárea como la biología molecular o la física de partículas.
    \item[Idioma] Aunque la mayoría de las revistas científicas están en inglés,
        hay revistas en otros idiomas como el español o el francés.
        Cabe mencionar que el inglés es el idioma \emph{de facto} de la
        ciencia.
        Publicar en otro idioma hoy por hoy limita la difusión del artículo.
    \item[Acceso abierto] Muchas revistas científicas cobran por el acceso a sus
        artículos (carísimo, por cierto).
        Sin embargo, en años recientes han surgido movimientos intelectuales
        que promueven al acceso abierto a la información científica, artística,
        y tecnológica bajo el principio de que el conocimiento es un bien
        público.
        Algunas revistas científicas solicitan a los autores que paguen por
        publicar sus artículos, lo que permite que los artículos sean de acceso
        abierto para cualquier lector.
    \item[Impacto] La calidad de una revista se mide por el número de veces que
        sus artículos son citados por otros científicos en sus propios trabajos.
        El \terminology{factor de impacto} se calcula dividiendo el número de
        citas que recibe la revista entre el número de artículos publicados en
        un periodo de tiempo.
        La idea de esta métrica es que una revista que publica artículos de alta
        calidad (creíbles, originales, y bien escritos) será citada con mayor
        frecuencia.
\end{description}

\subsection*{La revisión por pares}
Antes de publicar un artículo, este pasa por varias etapas de revisión que son
realizadas por otros colegas (llamados \emph{pares} en este contexto).
En principio, este proceso garantiza que los artículos publicados en la
revista sean de alta calidad.
Generalmente, las revistas de mayor prestigio y factor de impacto tienen un
proceso de revisión más riguroso.

La primera etapa de revisión es la evaluación por el editor, quien decide si el
artículo es relevante para la revista, si contribuye al conocimiento en el área,
y si cumple con las especificaciones técnicas de la revista como el formato y
la extensión.

En la segunda etapa se realiza una revisión externa, donde el editor envía el
manuscrito a dos o tres revisores expertos en el tema del artículo.
Esta revisión es anónima, es decir, los revisores no saben quién escribió el
artículo que están revisando, y el autor no sabe quiénes son los revisores.
Esto evita que los revisores tengan sesgos a favor o en contra del autor.
Al final del proceso los revisores envían un informe al editor con sus
comentarios y recomendaciones; aquí se decide si el artículo se publica, si
necesita modificaciones, o si se rechaza por completo.

Cuando un artículo requiere modificaciones da inicio la tercera etapa de
revisión.
Aquí el autor y los revisores trabajan juntos de manera iterativa para mejorar
el artículo hasta que esté listo para su publicación.

\begin{remember}
    El proceso de \terminology{revisión por pares} permite filtrar y mejorar los
    artículos científicos antes de su publicación.
    Este proceso garantiza que los artículos publicados en una revista sean de
    alta calidad.
\end{remember}

Me gustaría decir que este proceso es perfecto, pero en la práctica no lo es.
A veces los revisores no entienden el artículo, exigen cambios innecesarios, o
incluso rechazan el artículo por razones personales.
Entre los cambios más comunes que piden los revisores están la adición de
referencias a artículos de los mismos revisores, sacando provecho de su posición
de poder.
Más aún, cuando el artículo aborda temas hiperespecializados que solo un puñado
de científicos en el mundo entienden, es difícil para el editor encontrar
revisores que sean expertos en el tema sin que haya un conflicto de interés o
sin que se comprometa el anonimato de las partes involucradas.

Asimismo, tanto el editor como los revisores son humanos que dependen de su
experiencia y conocimiento para evaluar un artículo.
Cuando un artículo presenta un descubrimiento aparentemente revolucionario, es
difícil para los revisores y el editor evaluar dicho descubrimiento.
Por ejemplo, en 1981 la entonces candidata a doctora Mary M. Tai inventó un
modelo matemático para determinar el área bajo las curvas metabólicas; su método
fue tan popular entre sus colegas del hospital donde ella trabajaba que lo
apodaron la \emph{fórmula de Tai}.
En 1994 Tai publicó su modelo en un artículo científico de la revista
\emph{Diabetes Care}\cite{Tai1994}, donde el editor, los revisores externos,
los colaboradores de la autora (uno de ellos de la propia universidad de Yale),
y la autora misma, no detectaron lo absurdo de la situación: la fórmula de Tai
era simplemente la regla del trapecio; un método de integración numérica que se
enseña en el primer curso de cálculo de preparatoria o universidad.
El método es conocido desde la antigua Babilonia, fue usado por Arquímedes para
calcular el área de un círculo, y por Newton y Leibniz para desarrollar el
cálculo diferencial e integral.
Es una herramienta fundamental en ciencias e ingeniería y lo puedes consultar en
cualquier libro básico de cálculo como \cite{Arizmendi2018}.
Al momento de escribir estas líneas el artículo de Tai tiene al menos 540 citas,
siendo la más reciente de inicios de 2024.

Sin embargo, a pesar de estos problemas el proceso de revisión por pares es el
mejor que tenemos para garantizar la calidad de los artículos científicos, y
este tipo de problemas son menores en comparación con los beneficios que aporta.

\subsection*{El caso Wakefield}
A veces el proceso de revisión por pares falla, y un artículo científico
deficiente es publicado con graves consecuencias para la sociedad.
En 1998 la revista \emph{Lancet} publicó un artículo del entonces médico Andrew
Wakefield que sugería una relación entre la vacuna triple viral y el autismo.
La noticia se difundió rápidamente, apareciendo en los medios de comunicación
de todo el mundo.
En Reino Unido, donde se llevó a cabo el estudio, la confianza en la vacuna
disminuyó rápidamente, lo que años más tarde resultó en un brote de sarampión
que afectó a miles de personas.

La revisión por pares no dectó anomalías en el estudio de Wakefield a pesar de
que la muestra era muy pequeña y no era estadísticamente significativa; más aún,
una afirmación tan seria como la de Wakefield requería de un estudio mucho más
riguroso.
Apenas un mes después de la publicación del artículo, múltiples estudios
incapaces de replicar los resultados de Wakefield refutaron su hipótesis; a
estos se les conoce como \emph{estudio de replicación}, y es una forma de
revisión por pares que ocurre posterior a la publicación.

El polémico artículo y los subsecuentes trabajos de Wakefield siguieron causando
polémica en la sociedad en general durante años hasta que, en 2004, luego de una
investigación exhaustiva por parte del periodista Brian Deer, se desenmascaró el
fraude\cite{Deerc1127}.
Wakefield había sido contratado dos años antes por el abogado de un grupo
antivacunas que buscaba demandar a los fabricantes de la vacuna triple viral.
Como resultado, Wakefield alteró los datos del estudio seleccionando a 12 niños
que ya tenían autismo y haciendo que pareciera que la vacuna era la causa.
Asimismo se descubrió que el artículo no contaba con la aprobación de un comité
de ética y que los datos habían sido alterados.
A consecuencia de esto diez de los coautores del artículo se retractaron y
finalmente, en 2010, el Consejo General de Medicina del Reino Unido retiró la
licencia de Wakefield al tiempo que la revista \emph{Lancet} retiró el artículo
de su sitio web.

El caso Wakefield es un ejemplo de cómo la revisión por pares no es perfecta y
su impacto negativo en la sociedad cuando no se lleva a cabo de manera
adecuada.
Varias enfermedades y lesiones permanentes podrían haberse evitado si el
artículo de Wakefield no hubiera sido publicado, y es un recordatorio de la
responsabilidad que tienen los científicos y editores al publicar un artículo.

\section{Cómo leer un artículo científico}
\label{sec:comoleer}

Si te encuentras en un posgrado o trabajando en un laboratorio, es probable que
tengas que leer artículos científicos con regularidad.
Realmente no te recomiendo que leas un artículo científico sin una razón
específica, ya que son textos densos y especializados.
Pregúnate a ti mismo qué es lo que quieres o necesitas aprender; ¿quieres saber
cómo funciona un método concreto? ¿quieres conocer los resultados de un estudio
del cuál has escuchado hablar? ¿quieres conocer las novedades en tu área de
estudio? O si alguien te ha encargado leer un artículo, ¿qué es lo que esa
persona espera que aprendas?

\subsection*{Dónde encontrar artículos científicos}
Si aún no tienes un artículo en mente, hay varias maneras de encontrar uno.
Existen bases de datos de artículos científicos donde puedes buscar artículos
por palabras clave, autor, o revista.
Generalmente tu institución educativa o laboral tiene acceso a estas bases de
datos a través de su biblioteca, por lo que no necesitas pagar por el acceso a
los artículos.

Google ofrece una herramienta orientada a la búsqueda de artículos científicos
llamada \emph{Google Scholar}\footnote{\href{https://scholar.google.com/}%
    {\url{https://scholar.google.com/}}}.
Cada autor puede crear un perfil en Google Scholar donde se enlistan todos sus
artículos publicados, y cada artículo tiene un enlace a la versión en PDF o HTML
si está disponible.
Además, Google Scholar tiene herramientas para citar artículos, buscar artículos
relacionados, y crear alertas para recibir notificaciones cuando se publique un
artículo nuevo que cumpla con ciertos criterios de búsqueda.
En general, es una herramienta muy útil para la revisión de la literatura.

Si no tienes acceso institucional a las revistas científicas, hay varias
alternativas gracias al movimiento de acceso abierto.
El CONRICyT\footnote{\href{https://www.conricyt.mx/}%
    {\url{https://www.conricyt.mx/}}}, por ejemplo, ofrece un catálogo de
revistas de acceso abierto a las que puedes acceder desde cualquier computadora.
Sólo tienes que acceder a la sección \emph{recursos abiertos} de su sitio web y
seleccionar la editorial o base de datos que te interese.

Una de las bases de datos más interesantes la patrocina la Universidad de
Cornell, se llama \emph{arXiv}\footnote{\href{https://arxiv.org/}%
    {\url{https://arxiv.org/}}}, pero es un caso especial.
arXiv no es una revista ni tiene editoriales, sino que  es un repositorio de
manuscritos que pueden o no haber sido publicados en una revista científica.
En particular, no tiene un proceso de revisión por pares, no cobra por publicar
manuscritos, y tampoco cobra por acceder a ellos.
La única moderación que tiene arXiv es realizada por un equipo de voluntarios
que se encargan de verificar que los manuscritos cumplan con las normas de
arXiv, pero no verifican la calidad del contenido.
Celebridades de la ciencia como Stephen Hawking, Paul Ginsparg, y Juan Maldacena
han publicado manuscritos en arXiv.

\section{Cómo escribir un artículo científico}
\label{sec:comoescribir}